\chapter{Podsumowanie}\label{cha:podsum}

W ramach pracy napisana została aplikacja, która implementuje metodę źródeł pozornych w aplikacji na heterogeniczne platformy obliczeniowe. Wyniki testów wydajności z rozdziału 5.3 wykazały, że obliczenia algorytmu zaimplementowanego w~OpenCL wykonane z użyciem kart graficznych są znacznie szybsze niż na zwykłych procesorach i~możliwe jest skrócenie czasu obliczeń tej metody wykorzystując bardziej złożone platformy. Algorytm napisany w~OpenCL na procesorze CPU wykonuje się szybciej niż algorytm napisany w~czystym języku C++ dla rzędu odbić powyżej 11. Przy małej ilości odbić na szybkość algorytmu wpływa głównie czas dostępu do pamięci, który jest znacznie krótszy przy jednowątkowych aplikacjach.    

Wykorzystanie popularnej biblioteki OpenCL i~prezentacja aplikacji w~postaci otwartego kodu umożliwia wykorzystanie jej w~innych aplikacjach  lub do niezależnych badań. Otwarty kod kernela pozwala na implementacje programu w~dowolnym języku platformy obsługiwanym przez standard OpenCL (m. in. Python, Matlab).

Obliczenia w~rozdziale 5.2 ukazują użyteczność aplikacji w~analizie pomieszczeń. Analiza wyznaczonych echogramów pozwala na wstępną ocenę warunków akustycznych pomieszczenia i~daje informacje o wczesnych odbiciach. Użycie zewnętrznej nakładki graficznej w~programie GeoGebra pozwoliło na analizę poszczególnych odbić, co może być użyteczne przy projektowaniu rozmieszczenia ustrojów akustycznych w~pomieszczeniu. Wyznaczone siatki źródeł pozornych mogą posłużyć do dalszej analizy warunków akustycznych i~wyznaczenia wskaźników C50, C80, D50, wskaźnika zrozumiałości mowy i~innych parametrów nie rozpatrzonych w~pracy.