\chapter{Obliczenia testowe}\label{cha:ot}


%---------------------------------------------------------------------------

\section{Wprowadzenie}\label{sec:wprowadzenie}


Poza realizacją algorytmu autor wykonuje testy napisanej aplikacji. Testy poprawności implementacji zostały wykonane na podstawie testów jednostkowych. Przy użyciu ów testów autor sprawdza poprawność przekształceń geometrycznych i transferu danych między pamięciami urządzeń. Test poprawności całego modelu wykonano wyznaczając algebraicznie siatki źródeł pozornych dla kilku początkowych rzędów dla różnych układów geometrycznych pomieszczeń i porównując je z wynikami uzyskanymi przez obliczenia przy użyciu aplikacji. Testy dla źródeł pozornych wyższych rzędów zostały wykonane dla pojedynczych punktów.

W rozdziale 5.2 autor przedstawia przykładowe wykorzystanie aplikacji do analizy pola akustycznego. W tym celu przygotowane zostały geometryczne modele pomieszczeń o zróżnicowanych wymiarach i parametrach pochłaniania dźwięku, a następnie  porównane ze sobą.

W celu weryfikacji jednego z założeń pracy dotyczącego wydajności aplikacji autor uruchamia program na różnych urządzeniach. Do testów wykorzystane zostały urządzenia  zróżnicowane pod względem wydajności i ilości jednostek arytmetyczno-logicznych. Autor porównuje szybkość obliczeń na danych urządzeniach dla różnych rzędów siatki źródeł pozornych. 

%---------------------------------------------------------------------------
\section{Użycie aplikacji}\label{sec:asdasd}

\subsection{Obliczenia na pomieszczeniach zróżnicowanych geometrycznie}\label{sec:imstest1}

Do obliczeń dla zróżnicowanych pomieszczeń zdefiniowano 3 modele geometryczne (Tabele 5.1-5.3).

\begin{table}[h]
        \centering
        \begin{threeparttable}
                \caption{Geometryczny model pomieszczenia 1}\label{tab:table_example}
                \begin{tabularx}{0.6\textwidth}{| c | X | X | X |}
                        \midrule
                        		&	x & y & z \\
		Wymiary pomieszczenia & 6 & 10 & 4.6 \\
                        Źródło & 0.2 & 0.4 & 0.33 \\
		Punkt obserwacji & 0.6 & -1.7 & 0.4 \\
                        \bottomrule
                \end{tabularx}
        \end{threeparttable}
\end{table}

\begin{table}[h]
        \centering
        \begin{threeparttable}
                \caption{Geometryczny model pomieszczenia 2}\label{tab:table_example}
                \begin{tabularx}{0.6\textwidth}{| c | X | X | X |}
                        \midrule
                        		&	x & y & z \\
		Wymiary pomieszczenia & 6 & 10 & 4.6 \\
                        Źródło & -2.9 & -4.9 & -2.2 \\
		Punkt obserwacji & -2.87 & -4.39 & -2.08 \\
                        \bottomrule
                \end{tabularx}
        \end{threeparttable}
\end{table}

\begin{table}[h]
        \centering
        \begin{threeparttable}
                \caption{Geometryczny model pomieszczenia 3}\label{tab:table_example}
                \begin{tabularx}{0.6\textwidth}{| c | X | X | X |}
                        \midrule
                        		&	x & y & z \\
		Wymiary pomieszczenia & 20 & 2 & 2 \\
                        Źródło & -9 & 0.2 & 0.3 \\
		Punkt obserwacji & 8 & -0.4 & 0.1 \\
                        \bottomrule
                \end{tabularx}
        \end{threeparttable}
\end{table}

Dla danych modeli zostały wykonane wyliczenia siatek źródeł pozornych do 12 rzędu (rysunek 15-17).

Siatka 1

Siatka 2

Siatka 3

Na podstawie siatek źródeł pozornych zostały wyznaczone echogramy oraz krzywe zaniku dźwięku (rysunek 18-23).

Dużo rysunków


%---------------------------------------------------------------------------

\subsection{Obliczenia na pomieszczeniach o różnych parametrach pochłaniania}\label{sec:imstest2}

Do obliczeń dla zróżnicowanych współczynników pochłaniania wykorzystano model geometryczny 1 z rozdziału 5.2.1. Dla danego modelu zdefiniowano 3 różne zestawy współczynników pochłaniania (Tabela 5.4).

\begin{table}[h]
        \centering
        \begin{threeparttable}
                \caption{Wartośći współczynników pochłaniania dźwięku dla poszczególnych powierzchni w różnych zestawach danych}\label{tab:table_example}
                \begin{tabularx}{0.6\textwidth}{| c | X | X | X |}
                        \toprule
                        	powierzchnia &	zestaw 1 & zestaw 2 & zestaw 3 \\
                       \midrule
		góra & 0.71 & 0.21 & 0.71 \\
                        dół & 0.78 & 0.18 & 0.78 \\
		lewo & 0.85 & 0.25 & 0.02 \\
                     prawo & 0.72 & 0.12 & 0.72 \\
		przód & 0.84 & 0.24 & 0.84 \\
                    tył & 0.61 & 0.21 & 0.01 \\
                        \bottomrule
                \end{tabularx}
        \end{threeparttable}
\end{table}

Dla danych zestawów siatki źródeł pozornych stanowią te same punkty. Zróżnicowane będą jedynie poziomy poszczególnych promieni dźwiękowych dochodzących do punktu obserwacji, co można zauważyć na echogramach (rysunek 24-26).

Rysunki

Różne parametry pochłaniania mają wpływ na kształt i szybkość zanikania krzywej zaniku dźwięku (rysunki 27-29).

Rysowando

%---------------------------------------------------------------------------

%\subsection{Wizualizacja poszczególnych promieni dźwiękowych}\label{sec:imstest2}

%Podczas testów aplikacji użyteczna okazała się możliwość wizualizacji ścieżki wybranego promienia dźwiękowego. Aplikacja autora pracy umożliwia wygenerowanie skryptu jako plik wejściowy do programu GeoGebra, który umożliwia prezentację ścieżki promienia dźwiękowego dla wybranej wariacji powierzchni odbijających (rysunek).

%rysunek

%rysunek

%---------------------------------------------------------------------------

\section{Testy wydajnościowe}\label{sec:asdas2d}

Testy wydajnościowe przeprowadzono na 2 różnych architekturach procesorów – CPU i GPU. Do pomiarów na CPU posłużył procesor Intel Core i5-2520M (Tabela 5.5). Pomiaru przy użyciu GPU przeprowadzono na kartach Radeon R7 250X (Tabela 5.6) oraz Radeon R9 270X (Tabela 5.7).
 
\begin{table}[h]
        \centering
        \begin{threeparttable}
                \caption{Dane techniczne procesora Intel Core i5-2520M}\label{tab:table_example}
                \begin{tabularx}{0.6\textwidth}{| c | X |}
                       \midrule
		taktowanie & 2,50 GHz \\
                     liczba rdzeni & 2 \\
                    liczba wątków & 4 \\
                        \bottomrule
                \end{tabularx}
        \end{threeparttable}
\end{table}

\begin{table}[h]
        \centering
        \begin{threeparttable}
                \caption{Dane techniczne karty graficznej Radeon R7 250x}\label{tab:table_example}
                \begin{tabularx}{0.6\textwidth}{| c | X |}
                       \midrule
		taktowanie & 1000 MHz \\
                     liczba rdzeni GPU & 512 \\
                        \bottomrule
                \end{tabularx}
        \end{threeparttable}
\end{table}

\begin{table}[h]
        \centering
        \begin{threeparttable}
                \caption{Dane techniczne karty graficznej Radeon R9 270x}\label{tab:table_example}
                \begin{tabularx}{0.6\textwidth}{| c | X |}
                       \midrule
		taktowanie & 1030 MHz \\
                     liczba rdzeni GPU & 2560 \\
                        \bottomrule
                \end{tabularx}
        \end{threeparttable}
\end{table}

Pomiary przeprowadzono na modelu 1 z rozdziału 5.2.1 (rysunek aaaa) dla rzędów źródeł pozornych od 2 do 12. Wyniki zostały zestawione na poniższym wykresie (rysunek).

Wykres wydajności
















